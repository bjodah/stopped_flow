\section{Eyring-Teori}
\label{sec:eyring}
I laborationen utgår vi ifrån Arrhenius ekvation. I detta avsnitt
skall vi titta närmare på Eyrings ekvation (även känd som Eyring-Polanyi ekavtionen)
för en bimolekylär rekation:

\begin{align}
  \label{eq:ts}
  \ce{A + B} & \ce{<=> TS -> P} \\
  \label{eq:dpdt-propto}
  \frac{\mathrm{d}[\ce{P}]}{\mathrm{d}t} &\propto [TS]
\end{align}

Inom Eyring-teorin gör vi ett antagande om en mycket snabb
jämviktsinställning (steady-state) av vårt TS:

\begin{equation}
  \label{eq:ts-conc}
  [\ce{TS}] = \frac{[A][B]}{c^\plimsoll}K^\ddagger
\end{equation}

Från statistisk termodynamik kan vi härleda ett uttryck för
hastighetskonsanten för totalreaktionen:

\begin{align}
  \ce{A + B ->[k_2] P}
\end{align}

där $k_2$ då ges av:

\begin{align}
  \label{eq:eyring}
  k_2 &= \kappa \frac{k_\mathrm{B} T}{c^\plimsoll h} e^{\frac{\Delta S^\ddagger}{R}} e^{-\frac{\Delta H^\ddagger}{RT}}
\end{align}

där $\kappa$ är ``transmissionskoefficienten'' och antas ofta till 1.
Jämför denna med Arrhenius ekvation:

\begin{align}
  \label{eq:arrhenius}
  k_A &= A e^{-\frac{E_\mathrm{a}}{RT}}
\end{align}

Notera att \cref{eq:eyring,eq:arrhenius} inte är ekvivalenta med avseende
på temperaturberoende. 

Om vi antar att $E_a \equiv \Delta H^\ddagger$ kan vi för ett litet
temperaturintervall skriva:
\begin{align}
  A \approx& \frac{k_\mathrm{B} T}{c^\plimsoll h} e^{\frac{\Delta S^\ddagger}{R}}\\
  \frac{\Delta S^\ddagger}{R} \approx& \ln\left(\frac{c^\plimsoll h}{k_\mathrm{B}}
                                       \frac{A}{T}\right) 
\end{align}

Eftersom Eyring's ekvation är termodynamiskt härledd kan vi formulera ett
uttryck för bakåtreaktionens hastighetskonstant (ifall vi känner $\Delta H$ och $\Delta S$
för reaktionen):

\begin{align}
  k_b &= \kappa \frac{k_\mathrm{B} T}{h} e^{\frac{\Delta S^\ddagger - \Delta S}{R}}
        e^{-\frac{\Delta H^\ddagger - \Delta H}{RT}}
\end{align}

%%% Local Variables:
%%% mode: latex
%%% TeX-master: "../eyring"
%%% ispell-local-dictionary: "swedish"
%%% End:
