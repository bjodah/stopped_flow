\subsection{Spektrofotometri}
Spektrofotometern registrerar transmissionen $T$:

\begin{equation}
  T = I/I_0
\end{equation}
där $I$ är itensiteten av den strålen som går genom vårt prov
och $I_0$ är intensiteten av strålen som går genom vårt referensprov.
Absorbansen $A$ ges enligt:

\begin{equation}
  \label{eq:absorbance}
  A = -\log_{10}T
\end{equation}

Lambert-Beer's lag ger ett linjärt förhållande mellan absorbans och
koncentrationen av det absorberande ämnet (i vårt fall vårt färgade jon-komplex). 

\begin{equation}
  \label{eq:lambert-beer}
  A = \epsilon c l
\end{equation}

där $A$ är absorbansen, $\epsilon$ extinktionskoefficienten vid den valda
våglängden och $l$ kyvettlängden och $c$ concentrationen av det
absorberande ämnet. 
