\section{Övningsuppgift för dataanalys}
\label{sec:exercise}
Antag att ni upptäckt följande exotiska reaktion:

\begin{align}
  \label{eq:fourth-order}
  \ce{3A + B ->[k_f] C}
\end{align}

Reaktionen antas vara av tredje ordningen med avseende på A och första
ordningen med avseende på B. Ämne A är starkt färgat medan B och C är
färglösa.

Ni genomför ett antal experiment där ni varierar halten B enligt
följande: \SI{0.1}{\Molar}, \SI{0.2}{\Molar}, \SI{0.3}{\Molar}, \SI{0.4}{\Molar},
\SI{0.5}{\Molar}. Utgångshalten av A är alltid \SI{1}{\milli\Molar}.

För varje halt av B upprepar ni försöket 10 gånger. Under varje försök
antecknar ni vid 42 olika tidpunkter värden för absorbansen på lösningen.
Ni märker att ett par av era serier innehåller väldigt höga brusnivåer
(ni misstänker någon form av elektrisk störning).

Er uppgift är nu att bestämma hastighetskonstanten $k_f$. Mätdata för
analysen hittar ni i filen {\tt data.zip}. Filen innehåller fem mappar
(mappnamn är [B]$_0$ i molar) med vardera tio filer (replikat). Filerna
innehåller två kolumner: tid (i sekunder) och absorbans (enhetslös).

%%% Local Variables:
%%% mode: latex
%%% TeX-master: "../exercise"
%%% ispell-local-dictionary: "swedish"
%%% End:
