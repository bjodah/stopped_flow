kan utifrån
statistisk termodynamik och antagandet om ett transition state
\begin{align}
  \label{eq:K_TS}
  K_{TS} &=\frac{[TS]}{[A][B]}=e^{-\frac{\Delta G^{\ddag}}{RT}} \\
  \label{eq:k_tst}
  k_{reaction} &=\kappa\frac{k_{B}T}{h}K_{TS}
\end{align}

där $\kappa$ är den så kallade transmissionkoefficienten sätts ofta till
1 (ibland 0.5) och betecknar hur stor andel av de molekyler som når TS
fortsätter och bildar produkt.

Gibbs fria energi är:
\begin{equation}
  \label{eq:gibbs}
  \Delta G = \Delta H - T\Delta S
\end{equation}

strikt är $\Delta H$ inte en konstant under reaktionens gång utan
entalpin påverkas av volymsförändiringen.
\begin{equation}
  \label{eq:enthalpy}
  \Delta H = \Delta U + P\Delta V
\end{equation}
För reaktioner i vätskefas är dock detta bidrag väldigt litet (storleks
ordningen \SI{0.1}{\kilo\joule\per\mole}) och kan därmed förbises.
