\sectionlang{sv}{Laborationsrapport}
\sectionlang{en}{Report}
\label{sec:rapport}
\lang{sv}{
Laborationsrapporten skall skickas till laboratorieassistenten två
studieveckor (10 helgfria dagar utanför tenta-veckor) efter laborationens genomförande. Strukturen på
laborationsrapporten ska följa den konventionella disposition som ges i 
\cref{sec:rapport-disposition}.

\subsection{Disposition}
\label{sec:rapport-disposition}
Nedan följer en lista över rubriker och vad som hör till dem för er
laborationsrapport. 
\begin{description}
  \item[Titelsida] \hfill \\
    Namn på laboration, kurs, assistent, laboranter (inkl. epost) samt datum.
  \item[Sammanfattning] \hfill \\
    Vad som på engelska är känt som ``abstract''. Här sammanfattar ni i
    ett stycke vad som gjordes och anger de viktigaste resultaten
    (rapportera centrala kvantitativa resultat med eventuella
    konfidensintervall).
  \item[Inledning] \hfill \\ 
    Bakgrund till laborationen.
  \item[Teori] \hfill \\ 
    Använda teoretiska samband och eventuella härledningar.
  \item[Metod] \hfill \\ 
    Experimentell metod, experimentella förhållanden och analysförfarande.
  \item[Resultat och diskussion] \hfill \\ 
    De data ni erhållit från era analyser. Enskilda värden kan
    presenteras i flytande text, serier av värden förslagsvis i tabeller.
    Vid regressionsanalys kan en figur vara belysande. Ifall ni har
    ett stort antal figurer kan merparten av dem läggas i
    bilaga. Diskutera era resultat utifrån vad ni förväntade er och de
    approximationer ni gjort. I diskussionen finns ett visst utrymme för
    spekulationer.
  \item[Slutsatser] \hfill \\ 
    Sammanfatta det viktigaste från era resultat. Vad kan man med säkerhet
    säga (inga spekulationer)? Vad är (om något) fortfarande oklart och i
    behov av mer undersökning?
  \item[Referenser] \hfill \\
    Referenser till data/teori ni själva inte bestämt/härlett.
  \item[Bilaga - kod för analys] \hfill \\
    All kod för databehandling, kurvanpassing, figurgenerering (exempelvis
    Matlab skript).
  \item[Övriga bilagor] \hfill \\
    Exempelvis figurer för respektive mätserie med både rådata och
    passade kurvor. Passade parametrars värden i tabellform eller i
    etiketterna.
\end{description}
}
\lang{en}{
The laboratory report shall be sent to the laboratory assistant two
Study vouchers (10 week-long days outside the tent weeks) after the completion of the laboratory. The structure of
The laboratory report should follow the conventional disposition given in
\cref{sec:report-disposition}.

\subsection{Disposition}
\label{sec:report-disposition}
Below is a list of headlines and what belongs to them for you
Lab report.
\begin{description}
  \item [Title Page] \hfill \\
    Name of laboratory, course, assistant, laboratory staff (incl. Email) and date.
  \item [Summary] \hfill \\
    What is known in English as `` abstract ''. Here you summarize in
    A piece of what was done and indicate the most important results
    (Report central quantitative results with any
    Confidence interval).
  \item [Introduction] \hfill \\
    Background to the laboratory.
  \item [Theory] \hfill \\
    Use theoretical relationships and possible deductions.
  \item [Method] \hfill \\
    Experimental method, experimental conditions and analysis.
  \item [Result and discussion] \hfill \\
    The data you obtained from your analyzes. Individual values can
    Presented in floating text, series of values suggested in tables.
    In regression analysis, a figure may be illustrative. If you have
    A large number of figures can most of them be added
    Annex. Discuss your results based on what you expected and they are
    Approaches you made. There is some room for discussion in the discussion
    Speculation.
  \item [Conclusions] \hfill \\
    Summarize the most important from your results. What can be done with certainty
    Say (no speculation)? What is (if anything) still unclear and in
    Need more investigation?
  \item [References] \hfill \\
    References to data / theory you are not determined / derived.
  \item [Appendix - Code of Analysis] \hfill \\
    All code for data processing, curve fitting, image generation (for example
    Matlab script).
  \item [Other attachments] \hfill \\
    For example, figures for the respective measurement series with raw data and
    Matched curves. Matched parameters in tabular form or i
    Labels.
\end{description}
}
%%% Local Variables:
%%% mode: latex
%%% TeX-master: "../main"
%%% ispell-local-dictionary: "swedish"
%%% End:
