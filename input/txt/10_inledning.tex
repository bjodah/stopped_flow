\section{Inledning}
\label{sec:inledning}
I denna laboration skall ni, i vattenlösning vid olika temperaturer och
jonstyrkor, studera hastigheten för bildning av tiocyanatojärn(III) från
järn(III)joner och tiocyanatjoner (\ce{SCN-}):

\begin{align}
  \label{eq:equilibrium}
  \ce{Fe^3+ + SCN- <=>>[k_f][k_b] FeSCN^2+} & \hspace{2em} %
    \log_{10} \beta_1 = \num{2.065(5)} %
\end{align}
Stabilitetskonstanten\footnote{Stabilitetskonstanter är en observerad
  jämviktskonstant för en komplexbildningsreaktion vid en specificerad
  jonstyrka. Däri skiljer de sig från ``vanliga'' jämviktskonstanter som
  konventionellt rapporteras extrapolerade till obefintlig jonstyrka. För
  vår jämvikt betyder det:
  \begin{math}
    K_1 =
    \frac{\gamma_\ce{FeSCN^{2+}}}{\gamma_\ce{Fe^{3+}}\gamma_\ce{SCN^-}}
    \beta_1 =
    \frac{\gamma_\ce{FeSCN^{2+}}}{\gamma_\ce{Fe^{3+}}\gamma_\ce{SCN^-}}
    \frac{[\ce{FeSCN^{2+}}]}{[\ce{Fe^{3+}}][\ce{SCN-}]} c^{\plimsoll} =
    \frac{\{\ce{FeSCN^{2+}}\}}{\{\ce{Fe^{3+}}\}\{\ce{SCN-}\}}
  \end{math}

Notera närvaron av $c^{\plimsoll}$ vilket gör jämviktskonstanten
  enhetslös (något de flesta slarvar med att skriva ut). Inom kemi är
  $c^\plimsoll$ konventionellt \SI{1}{mol/dm^3}.
} ($\beta_1 = \frac{k_f}{k_b}$) för
tiocyanatojärn(III) är från
Referens\cite{peintler_improved_2000} och gäller för $I =
\SI{0.1}{\mole\per\kg}$ vid \SI{25}{\degreeCelsius}.

\ce{FeSCN^2+} är starkt rödfärgat ($\lambda_{max}=\SI{480}{\nm};
~\varepsilon_{\SI{480}{\nm}} =
\SI{5148(2)}{\per\Molar\per\centi\metre}$)
\cite{peintler_improved_2000} medan reaktanterna är (relativt) färglösa. Det finns
komplex med fler än en tiocyanatjon, men så länge koncentrationen av
\ce{SCN-} är tillräckligt låg kan vi förbise dessa.

\subsection{Mål}
Målet med laborationen är att ni självständigt skall undersöka både
temperatur- och jonstyrkeberoendet av hastighetskonstanten för
komplexbildningsreaktionen. Slutligen skall ni redovisa era resultat och
slutsatser i en laborationsrapport som följer strukturen för en
vetenskaplig artikel.

\subsection{Förberedelser}
Hur ni väljer att analysera temperatur- och jonstyrkeberoende är
upp till er, men förslagsvis utgår ni från väletablerade
teorier som introducerats i denna kurs. Utöver val av teoretisk behandling
kommer ni själva få välja experimentella förhållanden vid era försök.

Detta betyder att denna laboration är av en mer utmanande karaktär och
kräver troligen att ni avsätter avsevärd tid innan laborationen för att
beräkna lämpliga intervall för koncentrationer, jonstyrka och
temperatur. Den teoretiska behandling ställer vissa krav på kvalitet (och
kvantitet) av insamlade data. Det är därför viktigt att ni, innan
laborationens början, har en beräknat vilka koncentrationer ni vill
använda (se \cref{sec:instudering}).

Innan laborationstillfället skall ni också ha gjort de
instuderingsuppgifter som finns i \cref{sec:instudering}. Skicka in
skriftliga svar på dessa senast tre dagar före laborationstillfället till
laboratorieassistenten (\mailtobjorn eller \mailtocamilla).

%%% Local Variables:
%%% mode: latex
%%% TeX-master: "../main"
%%% ispell-local-dictionary: "swedish"
%%% End:
