\sectionlang{sv}{Inledning}
\sectionlang{en}{Introduction}
\label{sec:inledning}
\lang{sv}{
I denna laboration skall ni, i vattenlösning vid olika temperaturer och
jonstyrkor, studera hastigheten för bildning av tiocyanatojärn(III) från
järn(III)joner och tiocyanatjoner (\ce{SCN-}):
}
\lang{en}{
In this laboratory you will, in water solution at different temperatures and temperatures
Ionic strengths, study the rate of formation of thiocyanato iron (III) from
Iron (III) ions and thiocyanate ions
  }

\begin{align}
  \label{eq:equilibrium}
  \ce{Fe^3+ + SCN- <=>>[\SYMkf][k_\mathrm{b}] FeSCN^2+} & \hspace{2em} %
    \log_{10} \beta_1 = \num{2.065(5)} %
\end{align}

\newcommand{\Ktab}{\begin{align}%
    K_1 =%
    \frac{\gamma_{\ce{FeSCN^{2+}}}}{\gamma_{\ce{Fe^{3+}}}\gamma_{\ce{SCN^-}}}%
    \beta_1 =%
    \frac{\gamma_{\ce{FeSCN^{2+}}}}{\gamma_{\ce{Fe^{3+}}}\gamma_{\ce{SCN^-}}}%
    \frac{[\ce{FeSCN^{2+}}]}{[\ce{Fe^{3+}}][\ce{SCN-}]} c^{\plimsoll} =%
    \frac{\left\{\ce{FeSCN^{2+}}\right\}}{\left\{\ce{Fe^{3+}}\right\}\left\{\ce{SCN-}\right\}}%
  \end{align}%
}

\lang{sv}{
Stabilitetskonstanten\footnote{Stabilitetskonstanter är en observerad
  jämviktskonstant för en komplexbildningsreaktion vid en specificerad
  jonstyrka. Däri skiljer de sig från ``vanliga'' jämviktskonstanter som
  konventionellt rapporteras extrapolerade till obefintlig jonstyrka. För
  vår jämvikt betyder det:\\
  \Ktab\\
  Notera närvaron av $c^{\plimsoll}$ vilket gör jämviktskonstanten
  enhetslös (något de flesta slarvar med att skriva ut). Inom kemi är
  $c^\plimsoll$ konventionellt $\SI{1}{mol/dm^3} \equiv \SI{1000}{mol/m^3}$.
} ($\beta_1 = \frac{\SYMkf}{k_\mathrm{b}}$) för
tiocyanatojärn(III) är från
Referens\cite{peintler_improved_2000} och gäller för $I =
\SI{0.1}{\mole\per\kg}$ vid \SI{25}{\degreeCelsius}.

\ce{FeSCN^2+} är starkt rödfärgat ($\lambda_{\mathrm{max}}=\SI{480}{\nm};
~\varepsilon_{\SI{480}{\nm}} =
\SI{5148(2)}{\per\Molar\per\centi\metre}$)
\cite{peintler_improved_2000} medan reaktanterna är (relativt) färglösa. Det finns
komplex med fler än en tiocyanatjon, men så länge koncentrationen av
\ce{SCN-} är tillräckligt låg kan vi förbise dessa.
}
\lang{en}{
Stability Constant\footnote{Stability Constants are observed
  Equilibrium constant for a complex formation reaction at a specified
  Ion strength. They distinguish themselves from `` common '' equilibrium constants like
  Conventionally reported extrapolated to non-existent ionic strength. For
  Our equilibrium means:
 \Ktab \\
  Note the presence of $c ^{\plimsoll} $ which makes the equilibrium constant
  Unassuming (something most bothering to print). Within chemistry is
  $C ^\plimsoll $ conventional $\SI{1}{mol / dm ^ 3}\equiv\SI{1000}{mol / m ^ 3} $.
} ($\beta_1 =\frac{\SYMkf}{k_\mathrm{b}} $) for
Tiocyanato iron (III) is from
Reference\cite{peintler_improved_2000} and valid for $I =
\SI{0.1}{\mole\per\kg} $ vid\SI{25}{\degreeCelsius}.

\ce{FeSCN^2+} is strongly reddish ($\lambda_{\mathrm{max}} =\SI{480}{\nm};
~\epsilon_{\SI{480}{\nm}} =
\SI{5148(2)}{\per\Molar\per\centi\metre} $)
\cite{peintler_improved_2000} while the reactants are (relatively) colorless. There are
complexes with more than one thiocyanation, but so long the concentration of
\ce{SCN-} is low enough, we can overlook these.
}

\subsectionlang{sv}{Mål}
\subsectionlang{en}{Goals}
\lang{sv}{
Målet med laborationen är att ni självständigt skall undersöka både
temperatur- och jonstyrkeberoendet av hastighetskonstanten för
komplexbildningsreaktionen. Slutligen skall ni redovisa era resultat och
slutsatser i en laborationsrapport som följer strukturen för en
vetenskaplig artikel.
}
\lang{en}{
The goal of the laboratory is that you will independently investigate
Temperature and ionic strength dependence of the velocity constant for
Complex formation reaction. Finally, you will report your results and
Conclusions in a laboratory report that follows the structure of one
scientific article.
}
\subsectionlang{sv}{Förberedelser}
\subsectionlang{en}{Preparations}
\lang{sv}{
Hur ni väljer att analysera temperatur- och jonstyrkeberoende är
upp till er, men förslagsvis utgår ni från väletablerade
teorier som introducerats i denna kurs. Utöver val av teoretisk behandling
kommer ni själva få välja experimentella förhållanden vid era försök.

Detta betyder att denna laboration är av en mer utmanande karaktär och
kräver troligen att ni avsätter avsevärd tid innan laborationen för att
beräkna lämpliga intervall för koncentrationer och jonstyrka.
Den teoretiska behandling ställer vissa krav på kvalitet (och
kvantitet) av insamlade data.

Innan laborationstillfället skall ni också ha gjort de
instuderingsuppgifter som finns i \cref{sec:instudering}. Skicka in
individuella skriftliga lösningar på dessa i pdf-format (ej handskrivet)
senast 1 studievecka (5 helgfria dagar utanför tenta-veckor) före
laborationstillfället till  laboratorieassistenten, glöm inte att ange namn
och födelsedatum.
}
\lang{en}{
The goal of the laboratory is that you will independently investigate
Temperature and ionic strength dependence of the velocity constant for
Complex formation reaction. Finally, you will report your results and
Conclusions in a laboratory report that follows the structure of one
scientific article.
}
%%% Local Variables:
%%% mode: latex
%%% TeX-master: "../main"
%%% ispell-local-dictionary: "swedish"
%%% End:
