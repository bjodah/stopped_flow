\section{Laborationsrapport}
\label{sec:rapport}
Rapport

\subsection{Disposition}
\begin{description}
  \item[Inledning] \hfill \\ 
    Bakgrund
  \item[Teori] \hfill \\ 
    Härledningar
    \item[Metod] \hfill \\ 
      Experimentell metod och analys
    \item[Resultat] \hfill \\ 
      Hastighetskonstanter, Arrhenius paramaterar
    \item[Diskussion] \hfill \\ 
      Besvara frågor och diskutera era resultat med
      utgångspunkt i tillförlitlighet med avseende på
      approximationer.
    \item[Referenser] \hfill \\
      Referenser till data ni själva inte bestämt i laborationen.
    \item[Bilaga - kod för regressionsanalys] \hfill \\
      Kod för kurvanpassing (exempelvis Matlab script).
    \item[Bilaga - Figurer och data från regressionsanalys] \hfill \\
      Figurer för de respektive temperaturerna där både rådata och
      passade kurvor visas. Passade parametrars värden i tabellform.
\end{description}

\subsection{Uppgifter}
\begin{enumerate}
\item I \cref{sec:rev_binary} redovisas tre primitiva ekvationer till
integralen. Visa och kommentera vilket antagande som gjorts med avseende på
initialkoncentrationerna samt hastighetskonstanterna.
\item Undersök hur mycket antagandet av pseudo första ordningens beteende
  påverkar $k_f$ vid era experimentella förhållanden (ni kan göra
  antagandet att reaktionen är irreversibel).
\end{enumerate}

%%% Local Variables:
%%% mode: latex
%%% TeX-master: "../main"
%%% End:
