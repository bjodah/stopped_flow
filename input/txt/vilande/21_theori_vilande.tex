\begin{equation}
  \label{eq:scn-rate}
  \frac{d[\ce{SCN-}]}{dt} = \SYMkb[\ce{FeSCN^2+}] - \SYMkf[\ce{Fe^3+}][\ce{SCN-}]
\end{equation}

Frågan är nu hur man skall bestämma $\SYMkf$. För det första behöver vi ett funktionsuttryck
för [\ce{FeSCN^2+}]. Enklast är kanske i detta skede konstatera att under antagandet att
vi inte har någon mängd \ce{FeSCN^2+} vid reaktionens start gäller:

\begin{equation}
  \label{eq:fescn-scn-rel}
  [\ce{FeSCN^2+}] = [\ce{SCN-}]_0 - [\ce{SCN-}]
\end{equation}

vilket låter oss fokusera på att lösa \cref{eq:scn-rate}.

Detta kan göras med olika grad av förenklingar. Låt oss studera dem nedan.


\subsection{Arrhenius ekvation}
En termisk reaktion\footnote{En termisk reaktion är en vars hastighet begränsas av den termiska
energin hos molekylerna. Ett motexemepel är diffusionskontrollerade
radikalreaktioner som har en försumbar reaktionsbarriär.} har en
hastigetskonstant som är exponentiellt beroende på den (reciproka) temperaturen.
aktiverat komplex) antas ha en hastighetskonstant enligt:

\begin{equation}
  \label{eq:arrhenius}
  k = e^{-\frac{E_a}{RT}}
\end{equation}

där $k$ är hastighetskonstanten, $E_a$ aktiveringsenergin, $R$ den
allmänna gaskonstanten och $T$ är den (absoluta) temperaturen.

\cref{eq:arrhenius} postulerades 1889 av Svante Arrhenius och är ett
empiriskt samband (den förklarar experimentella observationer men saknar
en underliggande härledning). Tolkningen är att $E_a$ ger
energiskillnaden mellan det ``aktiverade komplexet'' och reaktanterna för
en reaktion. Ju större denna energiskillnad är, desto mer termisk energi
krävs för att nå detta aktiverade komplex som följdaktligen omvandlas
till prokukter.

Den moderna uppföljaren till Arrhenius ekvation är den s.k. ``transition
state teorin'' som kan härledas från statistisk termodynamik och ger
upphov till en mycket snarlik ekvation för hastighetskonstantens
temperaturberoende. Men i denna laboration nöjer vi oss med att behandla
våra data enligt Arrhenius ekvation.

\subsection{Primär kinetisk salteffekt}



Laddade specis i en lösning kan stabiliseras av  


\subsection{Spektrofotometri}
Spektrofotometern registrerar transmissionen $T$:

\begin{equation}
  T = I/I_0
\end{equation}
där $I$ är itensiteten av den strålen som går genom vårt prov
och $I_0$ är intensiteten av strålen som går genom vårt referensprov.
Absorbansen $A$ ges enligt:

\begin{equation}
  \label{eq:absorbance}
  A = -\log_{10}T
\end{equation}

Lambert-Beer's lag ger ett linjärt förhållande mellan absorbans och
koncentrationen av det absorberande ämnet (i vårt fall vårt färgade jon-komplex). 

\begin{equation}
  \label{eq:lambert-beer}
  A = \epsilon c l
\end{equation}

där $A$ är absorbansen, $\epsilon$ extinktionskoefficienten vid den valda
våglängden och $l$ kyvettlängden och $c$ concentrationen av det
absorberande ämnet. 


%%%%%%%%%%%%%%%%%%%%%%%%5


kan utifrån
statistisk termodynamik och antagandet om ett transition state
\begin{align}
  \label{eq:K_TS}
  K_{TS} &=\frac{[TS]}{[A][B]}=e^{-\frac{\Delta G^{\ddag}}{RT}} \\
  \label{eq:k_tst}
  k_{reaction} &=\kappa\frac{k_{B}T}{h}K_{TS}
\end{align}

där $\kappa$ är den så kallade transmissionkoefficienten sätts ofta till
1 (ibland 0.5) och betecknar hur stor andel av de molekyler som når TS
fortsätter och bildar produkt.

Gibbs fria energi är:
\begin{equation}
  \label{eq:gibbs}
  \Delta G = \Delta H - T\Delta S
\end{equation}

strikt är $\Delta H$ inte en konstant under reaktionens gång utan
entalpin påverkas av volymsförändiringen.
\begin{equation}
  \label{eq:enthalpy}
  \Delta H = \Delta U + P\Delta V
\end{equation}
För reaktioner i vätskefas är dock detta bidrag väldigt litet (storleks
ordningen \SI{0.1}{\kilo\joule\per\mole}) och kan därmed förbises.
