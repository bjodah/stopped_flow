\subsubsection{Förenklingar}

I denna laboration har vi förbisett följande reaktioner som också förekommer
\begin{tabular}{ll}
\ce{FeSCN^2+ + SCN- <=> Fe(SCN)_2^+}  & K=\SI{4.9}{\per\Molar} \\
\ce{Fe(SCN)_2^+ + SCN- <=> Fe(SCN)3}  & K=\SI{5.1}{\per\Molar} \\
\ce{Fe^3+ + H2O <=> FeOH^2+ + H+}     & $\log K = \num{-2.19}$ \\
\ce{Fe^3+ + 2H2O <=> Fe2OH4^2- + 2H+} & $\log K = \num{-2.95}$ \\
\end{tabular}


hur stora dessa effekter är beror som man ser på [\ce{SCN-}] och pH.

\begin{verbatim}
stability constants Bahta Tabell III.4 Ref (92O) => 
K. Ozutsumi, M. Kurihara, T. Miyazawa and T. Kawashima, Anal. Sci., 8(4), 521 (1992).

lgB1 2.11(1)
lgB2 3.34(2)
lgB3 3.82(9)
lbB4 3.6(3)
lgB5 4.3(2)
lgB6 5.0(2)

DeltaH1 -5.6(2)
DeltaH2 -11.0(4)
DeltaH3 -18.9(6)

DeltaS1 22(6)
DeltaS2 30(10)
DeltaS3 10(20)


Kinetics:
=========
below_kinetics_1958:
   I=0.40 
   dx/dt = k1*z*y + k2*z*y/[H+]
   k1 = 127 \pm 10 \per\molar\per\second
   k2 = 20.2 \pm 2 \per\second
   \Delta H_1^\ddag = 13.0 \pm 1.4 kcal/mol
   \Delta S_1^\ddag = -5   \pm  5  e.u.
   \Delta H_2^\ddag = 20.2 \pm 1.4 kcal/mol
   \Delta S_2^\ddag = 15   \pm 5   e.u.
   


peintler_improved_2000:
   k ~= 108, eps ~= 5150 ()
\end{verbatim}