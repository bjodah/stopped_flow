\section{Dataanalys}
\label{sec:analys}
Skriv ett skript (i exmepelvis Matlab, se exempelskript i
\cref{sec:matlab-nonlinear} i Bilaga) för att passa funktionsuttryck 
till era rådata. Gör passningen för både ``pseudo första ordningens''
och andra ordningens behandling i enlighet med härledningarna i
\cref{sec:irrev_unary,sec:irrev_binary}. Nedan följer några tips:

\begin{enumerate}
\item Kom ihåg att det är ett visst
  avstånd mellan blandkammaren i stopped-flow-utrustningen och
  kyvetten vilket betyder att $t_0$ är en parameter beroende på
  flödeshastigheten ni åstadkommer vid blandingen av reaktantlösningarna.
\item Hastighetsutrycket för pseudo första ordningens reaktion kan
  linjäriseras genom att göra passningen mot logaritmen av
  absorbansen. OLS\footnote{Ordinary Least Squares} passningen är
  då analytisk (``closed form'') vilket gör den lämplig att använda sig av för
  att bestämma initialgissning till den icke-linjära (iterativa)
  passningen som behövs för ``andra ordningens'' behandling.\footnote{
  Ni kan även göra en icke-linjär passning för pseudo första ordningens
  uttryck,   utgåendes från otransformerade data, dock kommer det
  troligtvis endast ha en marginell effekt på erhållna parametrar.}
\item Det kan vara bra att låta en fri parameter skala
  extinktionskoefficienten vid passningen. Denna parameter kommer då även
  att kompensera eventuellt fel i absolut koncentration av \ce{SCN-}.
\item Glöm inte att rapportera korrigering för primär kinetisk salteffekt
  för era hastighetskonstanter.
\item För att underlätta analysen kan det vara fördelaktigt att studera
  temperaturberoendet vid en given jonstyrka och {\em vice versa}:
  jonstyrkeberoendet vid en given temperatur (förslagsvis rumstemperatur).
% \item Ni behöver inte göra separata passningar för det reversibla
%   fallet. Förfarandet skiljer sig endast i den efterföljande
%   ekvationslösningen (där jämviktskonstanten nu behövs för bestämning av
%   $k_f$).
\end{enumerate}

%%% Local Variables:
%%% mode: latex
%%% TeX-master: "../main"
%%% ispell-local-dictionary: "swedish"
%%% End:
