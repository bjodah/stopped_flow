\section{Dataanalys}
\label{sec:analys}
Skriv ett skript (i exmepelvis Matlab, se exempelskript i
\cref{sec:matlab-nonlinear} i Bilaga) för att passa funktionsuttryck till
era rådata. Notera att vi inte tillåter att ni kopierar in rådata från
textfilerna in i koden (det försvårar utskrift och försämrar
överskådligheten). För en dataserie skall ni undersöka effekten av
antagendet om irreversibilitet. Funktionsuttryck får ni från härledningarna
i \cref{sec:irrev_unary,sec:rev_unary,sec:irrev_binary,sec:rev_binary}.
När ni utvärderat vilket funktionsuttryck som ger lägst osäkerhet i
bestämningen av $\SYMkf$ kan ni använda det för analysen av övriga
mätserier.

För varje kombination av $T$ och $I$ kommer ni göra $n$ replikat. För
varje replikat får ni från kurvanpassningen en uppskattning av
hastighetskonstanten $k_i$ med en uppskattad osäkerhet
($\sigma_i$). Dessa behandlar ni sedan statistiskt genom att ta ett
viktat medelväre ($\bar{k}$) där ni använder den reciproka variansen som
vikt ($w_i$):
\begin{align}
  w_i &= \frac{1}{\sigma_i^2} \\
  \bar{k} &= \frac{\sum_i{w_i \cdot k_i}}{\sum_i{w_i}}
\end{align}
variansen ($D^2$) för det viktade medelvärdet ges då av:
\begin{align}
%  D^2 &= \frac{1}{\sum_i 1/\sigma_i^2}
   D^2 &= \frac{\sum_i w_i(k_i - \bar{k})^2}{(n - 1)\sum_j w_j}
\end{align}
vid rapportering av $\bar{k}$ anges osäkerheten konventionellt som en
standardavvikelse ($\sigma \approx \sqrt{D^2}$):
\begin{align}
  k_f &= \bar{k} \pm \sqrt{D^2}
\end{align}
vilket ger ett konfidensintervall på drygt 68\%.\footnote{
Mer finns att läsa här:
%\url{http://en.wikipedia.org/wiki/68\%E2\%80\%9395\%E2\%80\%9399.7_rule}
\url{http://en.wikipedia.org/wiki/68-95-99.7\_rule}
}. Efterföljande regressioner baserade på dessa
data gör ni då viktade med den reciproka variansen ($\frac{1}{D^2}$) som
vikt.

Nedan följer några tips:

\begin{enumerate}
\item Kom ihåg att det är ett visst
  avstånd mellan blandkammaren i stopped-flow-utrustningen och
  kyvetten vilket betyder att $t_0$ är en parameter beroende på
  flödeshastigheten ni åstadkommer vid blandingen av
  reaktantlösningarna. Ni behöver således införa den som en parameter i
  ekvationerna som används vid passningen.
\item Det kommer sannolikt behövas en fri parameter som skalar
  extinktionskoefficienten vid passningen (speciellt ifall ni använder en
  annan våglängd än $\lambda_{max}$). Denna parameter kommer då även att
  kompensera eventuellt fel i beräknad koncentration av \ce{FeSCN^{2+}}
  (vilket kan vara ganska stort pga. jonstyrkeeffekter och ev. antagande
  om irreversibilitet).
\item För att underlätta analysen kan det vara fördelaktigt att studera
  temperaturberoendet vid en given jonstyrka och {\em vice versa}:
  jonstyrkeberoendet vid en given temperatur (förslagsvis
  rumstemperatur). Att variera jonstyrka och temperatur samtidigt är
  förvisso intressant, men insamling av data och efterföljande analys
  blir för tidskrävande för denna laboration.
\item Initialgissning för hastighetskonstanten kan ni uppskatta visuellt
  från en plot av absorbans mot tid under antagandet irreversibel pseduo
  första ordningens reaktion. Då gäller: $\SYMz(\SYMt) = \SYMZ e^{-k' \SYMt} \implies k'
  \approx 1/T_{0.37}$ -- där $T_{0.37}$ är tiden det tar för 63\% av den
  begränsande reaktanten att försvinna.

% \item Hastighetsutrycket för pseudo första ordningens reaktion kan
%   linjäriseras genom att göra passningen mot logaritmen av
%   koncentrationen av den begränsande reaktanten. OLS\footnote{Ordinary
%     Least Squares} passningen är då analytisk (``closed form'') vilket
%   gör den lämplig att använda sig av för att bestämma initialgissning
%   till den icke-linjära (iterativa) passningen som behövs för ``andra
%   ordningens'' behandling.\footnote{
%   Ni kan även göra en icke-linjär passning för pseudo första ordningens
%   uttryck,   utgåendes från otransformerade data, dock kommer det
%   troligtvis endast ha en marginell effekt på erhållna parametrar.}

% \item Glöm inte att rapportera korrigering för primär kinetisk salteffekt
%   för era hastighetskonstanter.

% \item Ni behöver inte göra separata passningar för det reversibla
%   fallet. Förfarandet skiljer sig endast i den efterföljande
%   ekvationslösningen (där jämviktskonstanten nu behövs för bestämning av
%   $k_f$).
\end{enumerate}

%%% Local Variables:
%%% mode: latex
%%% TeX-master: "../main"
%%% ispell-local-dictionary: "swedish"
%%% End:
