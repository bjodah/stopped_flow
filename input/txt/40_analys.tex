\sectionlang{sv}{Dataanalys}
\sectionlang{en}{Data analysis}
\label{sec:analys}
\providecommand{\kbarequation}{\begin{align}%
  \bar{k} &= \frac{\sum_i{w_i \cdot k_i}}{\sum_i{w_i}}%
\end{align}%
}
\providecommand{\Dequation}{\begin{align}%
   D^2 &= \frac{\sum_i w_i(k_i - \bar{k})^2}{(n - 1)\sum_j w_j}%
\end{align}%
}
\providecommand{\kfequation}{\begin{align}%
  \SYMkf &= \bar{k} \pm \sqrt{D^2}%
\end{align}%
}
\lang{sv}{
Skriv ett skript (i exempelvis Matlab, se exempelskript i
\cref{sec:matlab-nonlinear} i Bilaga) för att passa funktionsuttryck till
era rådata. Notera att vi inte tillåter att ni kopierar in rådata från
textfilerna i någon annan fil (det försvårar utskrift och försämrar
överskådligheten), utan filerna skall läsas programmatiskt. För en
mätserie skall ni undersöka effekten av antagandet om
irreversibilitet. Funktionsuttryck får ni från härledningarna i
\cref{sec:irrev_unary,sec:rev_unary,sec:irrev_binary,sec:rev_binary}.
Ni måste undersöka effekten av reversibilitet, det är frivilligt ifall
ni även vill undersöka utrryck för andra ordningens reaktionskinetik.

%% När ni utvärderat vilket funktionsuttryck som ger lägst osäkerhet i
%% bestämningen av $\SYMkf$ kan ni använda det för analysen av övriga
%% mätserier.

För varje kombination av $T$ och $I$ kommer ni göra $n$ replikat. För
varje replikat får ni från kurvanpassningen en uppskattning av
hastighetskonstanten $k_i$. Dessa behandlar ni sedan statistiskt, helst
genom att ta ett viktat medelväre ($\bar{k}$):
\kbarequation
som vikt ($w_i$) brukar den reciproka variansen användas
($\sigma_i^{-2}$), eller något som korrelerar med den, ett vanligt val är
MAD (Mean Absolute Deviation). Variansen ($D^2$) för det viktade
medelvärdet ($\bar{k}$) ges då av:
\Dequation
vid rapportering av $\bar{k}$ anges osäkerheten konventionellt som en
standardavvikelse:
\kfequation
vilket ger ett konfidensintervall på drygt 68\%.\footnote{
Mer finns att läsa här:
%\url{http://en.wikipedia.org/wiki/68\%E2\%80\%9395\%E2\%80\%9399.7_rule}
\url{http://en.wikipedia.org/wiki/68-95-99.7\_rule}
}. Efterföljande regressioner baserade på dessa
data kan ni då göra med den reciproka variansen ($\frac{1}{D^2}$) som
vikt.

Nedan följer några tips:

\begin{enumerate}
\item Kom ihåg att det är ett visst
  avstånd mellan blandkammaren i stopped-flow-utrustningen och
  kyvetten vilket betyder att $t_0$ är en parameter beroende på
  flödeshastigheten ni åstadkommer vid blandingen av
  reaktantlösningarna. Ni behöver således införa den som en parameter i
  ekvationerna som används vid passningen.
\item Det kommer sannolikt behövas en fri parameter som skalar
  extinktionskoefficienten vid passningen (speciellt ifall ni använder en
  annan våglängd än $\lambda_\mathrm{max}$). Denna parameter kommer då även att
  kompensera eventuellt fel i beräknad koncentration av \ce{FeSCN^{2+}}
  (vilket kan vara ganska stort pga. jonstyrkeeffekter och ev. antagande
  om irreversibilitet).
\item För att underlätta analysen kan det vara fördelaktigt att studera
  temperaturberoendet vid en given jonstyrka och {\em vice versa}:
  jonstyrkeberoendet vid en given temperatur (förslagsvis
  rumstemperatur). Att variera jonstyrka och temperatur samtidigt är
  förvisso intressant, men insamling av data och efterföljande analys
  blir för tidskrävande för denna laboration.
\item Initialgissning för hastighetskonstanten kan ni uppskatta visuellt
  från en plot av absorbans mot tid under antagandet irreversibel pseduo
  första ordningens reaktion. Då gäller: $\SYMz(\SYMt) = \SYMZ e^{-k' \SYMt} \implies k'
  \approx 1/T_{0.37}$ -- där $T_{0.37}$ är tiden det tar för 63\% av den
  begränsande reaktanten att försvinna.\footnote{$T_{0.37}$ kan verka
    kryptisk, den kan ses som en analog till halveringstiden $T_{1/2}$ i
  $f(x)=2^{-\frac{t}{T_{1/2}}}$ men då basen är $e$ skriver vi uttrycket som $g(x)=e^{-\frac{t}{T_{0.37}}}$.}

% \item Hastighetsutrycket för pseudo första ordningens reaktion kan
%   linjäriseras genom att göra passningen mot logaritmen av
%   koncentrationen av den begränsande reaktanten. OLS\footnote{Ordinary
%     Least Squares} passningen är då analytisk (``closed form'') vilket
%   gör den lämplig att använda sig av för att bestämma initialgissning
%   till den icke-linjära (iterativa) passningen som behövs för ``andra
%   ordningens'' behandling.\footnote{
%   Ni kan även göra en icke-linjär passning för pseudo första ordningens
%   uttryck,   utgåendes från otransformerade data, dock kommer det
%   troligtvis endast ha en marginell effekt på erhållna parametrar.}

% \item Glöm inte att rapportera korrigering för primär kinetisk salteffekt
%   för era hastighetskonstanter.

% \item Ni behöver inte göra separata passningar för det reversibla
%   fallet. Förfarandet skiljer sig endast i den efterföljande
%   ekvationslösningen (där jämviktskonstanten nu behövs för bestämning av
%   $\SYMkf$).
\end{enumerate}
}
\lang{en}{
Write a script (for example, Matlab, see sample script in
\cref{sec:matlab-nonlinear} in Appendix) to fit function expressions to
Your raw data. Please note that we do not allow you to copy raw data from
Text files in any other file (it complicates printing and degrades
Transparency), without the files being programmatically read. For one
Measurement series, you will examine the effect of the assumption
Irreversibility. Functional expressions are derived from the deductions in
\cref{sec:irrev_unary,sec:rev_unary,sec:irrev_binary,sec:rev_binary}.
You must investigate the effect of reversibility, it is optional
You also want to investigate the impressions of the second order reaction kinetics.

%% When you evaluated which function expression gives the least uncertainty
%% determination of $ \ SYMkf $ can you use for the analysis of others
%% measurement series.

For each combination of $T$ and $I$ you will make $n$ replicate. For
Each replicate gives you an estimate of
Rate constant $k_i$. These are then treated statistically, preferably
By taking a weighted average ($\bar{k}$):
\kbarequation
As a weight ($w_i$), the reciprocal variance usually uses
($\sigma_i^{-2}$), or something that correlates with it, a common choice is
MAD (Mean Absolute Deviation). The variance ($D^2$) for the weighted
The mean value ($\bar{k}$) is then given by:
\Dequation

when reporting $\bar{k}$, the uncertainty is conventionally stated as one
Standard deviation:
\kfequation

Giving a confidence interval of just over 68\%.\footnote{
More to read here:
% \ Url {http://en.wikipedia.org/wiki/68\%E2\%80\%9395\%E2\%80\%9399.7_rule}
\url{http://en.wikipedia.org/wiki/68-95-99.7\_rule}
}. Subsequent regressions based on these
Data can you then do with the reciprocal variance ($\frac{1}{D ^ 2} $) as
weight.

Below are some tips:

\begin{enumerate}
\item Remember that there is a certain
  Distance between the mixing chamber in the stopped-flow equipment and
  Cuvette which means that $ t_0 $ is a parameter depending on
  The flow rate ni results in the mixing of
  Reactant solutions. You need to introduce it as a parameter in
  Equations used for the fit.
\item It is likely that a free parameter will be scaled
  Extinction coefficient at the fitting (especially if you use one
  Wavelength other than $\lambda_\mathrm{max}$). This parameter will also be that
  Compensate any errors in the calculated concentration of \ce{FeSCN^{2+}}
  (Which may be quite large due to ionic effects and possibly assumption
  On irreversibility).
\item To facilitate the analysis, it may be beneficial to study
  Temperature dependency at a given ionic strength and {\em vice versa}:
  Ionic strength dependence at a given temperature (preferably
  Room temperature). To vary the ionic strength and temperature at the same time is
  Certainly interesting, but data collection and subsequent analysis
  Becomes too time consuming for this lab.
\item Initial guess for the speed constant you can appreciate visually
  From a plot of absorbance to time during the assumption irreversible pseduo
  First order reaction. Then: $\SYMz(\SYMt) = \SYMZ e^{-k'\SYMt} \implies k'
  \ Approx 1 / T_ {0.37} $ - where $ T_ {0.37} $ is the time it takes for 63% of the
  Limiting reactant to disappear. \footnote{$ T_ {0.37} $ may look
    cryptic, it can be seen as an analog to the half-life $T_{1/2}$ in
  $f(x) = 2^{-\frac{t}{T_{1/2}}}$ but when the base is $e$
    we write the expression as $g(x) = e^{-\frac{t}{T_{0.37}}} $.}
\end{enumerate}

}
%%% Local Variables:
%%% mode: latex
%%% TeX-master: "../main"
%%% ispell-local-dictionary: "swedish"
%%% End:
