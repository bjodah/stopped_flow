\section{Teori}
\label{sec:teori}
Laborationen introducerar ingen ny teori. Föreläsningarna samt
kurslitteratur för kursen innehåller det ni behöver. Men som ledning för
den numeriska analysen ges här fullständiga härledningar av
koncentrationernas tidsberoende för olika grad av förenkling av
hastighetsuttrycken.

\subsection{Hastighetsuttryck}
För läslighetens skull betecknar vi från och med nu [\ce{FeSCN^2+}],
[\ce{Fe^3+}] och [\ce{SCN-}] med variablerna $\SYMx$, $\SYMy$ och $\SYMz$ och de
initiala koncentrationerna med respektive versaler $\SYMX$, $\SYMY$ och $\SYMZ$. Vi
kan då beskriva koncentrationerna vid tiden $\SYMt$ enligt:

%\begin{table}
\begin{center}
\begin{tabular}{rccccc}
        & \ce{Fe^3+} & + & \ce{SCN-} & \ce{<=>>[k_f][k_b]} & \ce{FeSCN^2+} \\
  $t=0$ &     $Y$    &   &    $Z$    &                    &      $X$        \\
  $t>0$ &  $y=Y+X-x$ &   & $z=Z+X-x$ &                    &      $x$        \\
\end{tabular}
\end{center}
%\end{table}

%%% Local Variables:
%%% mode: latex
%%% TeX-master: "../main"
%%% ispell-local-dictionary: "swedish"
%%% End:


Idealt har vi ingen produkt innan blandning av reaktantlösningarna
($\SYMX=0$). För läslighetens skull baserar vi följande härledningar på det
antagandet:

\begin{center}
\begin{tabular}{rccccc}
        & \ce{Fe^3+} & + & \ce{SCN-} & \ce{<=>>[\SYMkf][\SYMkb]} & \ce{FeSCN^2+} \\
  $\SYMt>0, \SYMX=0$ &  $\SYMy=\SYMY-\SYMx$ &   & $\SYMz=\SYMZ-\SYMx$ &                    &      $\SYMx$        \\
\end{tabular}
\end{center}

%%% Local Variables:
%%% mode: latex
%%% TeX-master: "../main"
%%% ispell-local-dictionary: "swedish"
%%% End:


Det är från dataserier för $\SYMx(t)$ (eller snarare absorbans mot tid) som
ni genom kurvanpassning bestämmer $\SYMkf$ för varje kombination av
temperatur och jonstyrka.

Man kan härleda ett explicit analytiskt uttryck för $\SYMx(t)$ genom att
integrera den ordinära differentialekvation som beskriver dess
tidsberoende. Reaktionen är av andra ordningen och reversibel, men vi kan
välja att beskriva denna med olika grad av förenklingar:
\begin{description}
\item[Irreversibel] \hfill \\
  Om en jämvikt är kraftigt förskjuten mot produkt kommer
  bakåtreaktionen vara av liten betydelse och kan därmed
  ignoreras i behandlingen av kinetiken för att få
  ett enklare hastighetsuttryck.
\item[Pseudo första ordningen] \hfill \\ %
  Ifall initialkoncentrationen av en reaktant är mycket större än den andra,
  kan reaktanten i överskott approximeras som konstant under reaktionens
  gång. Man bestämmer då istället den så kallade ``pseudo första ordningens''
  hastighetskonstant som implicit inkluderar den högre koncentrationen.
\end{description}

\begin{table}
  \caption[Hastighetsuttryck för tiocyanatojärn(III)]{Hastighetsuttryck
    ($\frac{d\SYMx}{dt}$) med olika grad av förenkling. Antagandet $\SYMY \gg \SYMZ $
  gäller för pseudo 1:a ordningens uttryck.} % [\ce{FeSCN^2+}]
  \label{tab:rate_eqs}
  \begin{center}
  \begin{tabular}{lll}
   \toprule
         {}
           &
         Irreversibel
           &
         Reversibel
       \tabularnewline
   \midrule
            Pseudo 1:a ordn.
               &
            $Y k_{f} \left(Z - x\right)$
               &
            $Y k_{f} \left(Z - x\right) - k_{b} x$
        \tabularnewline
             2:a ordn.
               &
            $k_{f} \left(Y - x\right) \left(Z - x\right)$
               &
            $- k_{b} x + k_{f} \left(Y - x\right) \left(Z - x\right)$
        \tabularnewline
   \bottomrule
  \end{tabular}
  \end{center}
  \footnotesize
\end{table}


I \cref{tab:rate_eqs} ges olika uttryck för $\frac{d\SYMx}{dt}$ där vi i
fallen för pseudo första ordningens uttryck antagit att $[\ce{Fe^3+}]_0
\gg [\ce{SCN-}]_0$. I
\cref{sec:irrev_unary,sec:rev_unary,sec:irrev_binary,sec:rev_binary}
följer härledningar för vart och ett av dessa fyra fall.


\subsubsection{Irreversibel pseudo första ordningens reaktion}
\label{sec:irrev_unary}
Denna är den enklaste modell vi kan ansätta (och även den med störst fel).
Jämviktskonstanten visar att den ``framåtgående'' reaktionen (bildandet av \ce{FeSCN^2+})
är den dominerande. Vidare antar vi att $[\ce{Fe^3+}]_0 \gg [\ce{SCN-}]_0$ vilket
betyder att koncentrationen järn(III) kan approximeras som konstant under
reaktionens gång. Om vi utnyttjar dessa antaganden får vi ett förenklat
uttryck för tidsutvecklingen av [\ce{SCN-}]:

\begin{equation}
  \label{eq:scn-pseudo-rate}
  \frac{d[\ce{SCN-}]}{dt} = -\SYMkf[\ce{Fe^3+}]_0[\ce{SCN-}] = -k'[\ce{SCN-}]
\end{equation}

där $k' = \SYMkf[\ce{Fe^3+}]_0$ och kallas för ``pseudo första ordningens
hastighetskonstant''. \cref{eq:scn-pseudo-rate} har en enkel lösning:

\begin{align}
  \frac{d}{d t} x = Y k_{f} \left(Z - x\right) \\
  \int_{0}^{x} \frac{1}{Y k_{f} \left(Z - \chi\right)}\, d\chi = \int_{0}^{t} 1\, d\tau \\
  \frac{1}{Y k_{f}} \left(\log{\left (Z \right )} - \log{\left (Z - x \right )}\right) = t \\
  x = Z \left(1 - e^{- Y k_{f} t}\right)
\end{align}


Notera att vi bytte symbol för integranden från $\SYMx$ till $\SYMchi$ för att
undvika förväxling med den övre integrationsgränsen (detsamma gäller
bytet från $\SYMt$ till $\SYMtau$).

\subsubsection{Reversibel pseudo första ordningens reaktion}
\label{sec:rev_unary}
Ett steg mot en mer korrekt beskrivning är att även beakta
bakåtreaktionen (vi betraktar fortfarande [\ce{Fe^3+}] som
konstant). Härledningen är analog och blir:
\begin{align}
  \frac{d}{d t} x = k'_{f} \left(Z - x\right) - k_{b} x \\
  \int_{0}^{x} \frac{1}{- \chi k_{b} + k'_{f} \left(Z - \chi\right)}\, d\chi = \int_{0}^{t} 1\, d\tau \\
  \frac{1}{k'_{f} + k_{b}} \left(\log{\left (- Z k'_{f} \right )} - \log{\left (- Z k'_{f} + x \left(k'_{f} + k_{b}\right) \right )}\right) = t \\
  x = \frac{Z k'_{f} \left(e^{t \left(k'_{f} + k_{b}\right)} - 1\right)}{\left(k'_{f} + k_{b}\right) e^{t \left(k'_{f} + k_{b}\right)}} \\
  x = \frac{Z k'_{f}}{k'_{f} + k_{b}} \left(1 - e^{- t \left(k'_{f} + k_{b}\right)}\right)
\end{align}

Kom ihåg att $\SYMkf$ och $\SYMkb$ inte är oberoende av varandra då de skall reproducera
stabilitetskonstanten i \cref{eq:equilibrium}.
% Notera att funktionsformen, $f(x) = C_1(1-e^{C_2x})$, är densamma
% som i \cref{sec:irrev_unary}.

\subsubsection{Irreversibel bimolekylär kinetik}
\label{sec:irrev_binary}
Om vi istället fokuserar på att bättre beskriva effekten av att
[\ce{Fe^3+}] är tidsberoende får vi en något svårare integral att lösa:

\begin{align}
  \frac{d}{d t} x = k_{f} \left(Y - x\right) \left(Z - x\right) \\
  \int_{0}^{x} \frac{1}{\left(Y - \chi\right) \left(Z - \chi\right)}\, d\chi = \int_{0}^{t} k_{f}\, d\tau
\end{align}

Den primitiva funktionen kan vi erhålla genom att slå upp den i ``BETA Handbook of
Mathematics'', integrera för hand med partialbråksuppdelning eller
använda ett CAS\footnote{  Computer Algebra System -   exempelvis
  Mathematica, Maple, Maxima, SymPy m.fl.}. Friställandet av vår beroende
variabel $\SYMx$ till ett explicit uttryck i den oberoende variabeln $\SYMt$
blir:

\begin{align}
  \frac{1}{Y - Z} \left(\log{\left (\frac{Z}{Y} \right )} + \log{\left (Y - x \right )} - \log{\left (Z - x \right )}\right) = k_{f} t \\
  x = \frac{Y \left(- e^{k_{f} t \left(- Y + Z\right)} + 1\right)}{\frac{Y}{Z} - e^{k_{f} t \left(- Y + Z\right)}}
\end{align}

vi ser att det detta explicita uttryck inte längre kan linjäriseras
vilket gör en regression mer avancerad.

\subsubsection{Reversibel bimolekylär kinetik}
Slutligen härleder vi den mest detaljerade beskrivningen av vårt kinetikproblem.
\label{sec:rev_binary}
\begin{align}
  \frac{d}{d t} x = - k_{b} x + k_{f} \left(Y - x\right) \left(Z - x\right) \\
  \int_{0}^{x} \frac{1}{- \chi k_{b} + k_{f} \left(Y - \chi\right) \left(Z - \chi\right)}\, d\chi = \int_{0}^{t} 1\, d\tau
\end{align}
\begin{align}
  \int \frac{1}{a x^{2} + b x + c}\, dx = \begin{cases} C + \frac{1}{\sqrt{- 4 a c + b^{2}}} \log{\left (\frac{2 a x + b - \sqrt{- 4 a c + b^{2}}}{2 a x + b + \sqrt{- 4 a c + b^{2}}} \right )} & \text{for}\: 4 a c < b^{2} \\C - \frac{2}{2 a x + b} & \text{for}\: 4 a c = b^{2} \\C + \frac{2}{\sqrt{4 a c - b^{2}}} \operatorname{atan}{\left (\frac{2 a x + b}{\sqrt{4 a c - b^{2}}} \right )} & \text{for}\: 4 a c > b^{2} \end{cases}
\end{align}
\begin{align}
  \begin{Bmatrix}a : k_{f}, & b : - Y k_{f} - Z k_{f} - k_{b}, & c : Y Z k_{f}\end{Bmatrix} \\
  P = \sqrt{- 4 a c + b^{2}} \\
  \frac{1}{P} \left(- \log{\left (\frac{- P + b}{P + b} \right )} + \log{\left (\frac{- P + 2 a x + b}{P + 2 a x + b} \right )}\right) = t \\
  x = \frac{\left(P - b\right) \left(- P e^{P t} + P - b e^{P t} + b\right)}{2 a \left(P + b + \left(P - b\right) e^{P t}\right)} \\
  Q = P + b \\
  R = P - b \\
  x = - \frac{Q \left(e^{P t} - 1\right)}{2 a \left(\frac{Q}{R} + e^{P t}\right)}
\end{align}

Notera att vi antog $4\SYMa\SYMc < \SYMb^2$ vilket går att visa, men lämnas som en
övning för läsaren.
% Notera att funktionsformen, $f(x) = C_1\frac{1 - e^{C_2x}}{C_3 -
%   e^{C_2x}}$, är densamma som i \cref{sec:irrev_binary}.


\subsection{Hastighetskonstanter}
Hastighetskonstanter är verkliga konstanter för en given temperatur och
jonstyrka. I era försök kommer dessa parametrar att variera och det är
upp till er att behandla dessa effekter enligt någon av de modeller som
behandlas i kursen.

\subsection{Jonstyrkeeffekter}
Jonstyrkan\footnote{Jonstyrka som begrepp introducerades av
  G. N. Lewis och M. Randall 1921, detta var en rent empirisk parameter
  som först i efterhand fick en teoretisk förklaring av Debye \&
  H\"uckel, den som är intresserad av detaljerna kan med fördel läsa
  Ref.\cite{sastre_de_vicente_concept_2004}
} för en lösning beräknas enligt:

\begin{equation}
  \label{eq:ionic-strength}
  I = \frac{1}{2}\sum_i b_iz_i^2
\end{equation}

där $I$ är jonstyrkan, $b_i$ är molaliteten av ämne $i$ och $z_i$ är
laddningen för jonen.

På liknande vis som polära lösningsmedel stabiliserar
solvatiserade joner kan åskådarjoner (joner som inte
direkt deltar i reaktionen) påverka stabiliteten av våra joner och
jonkomplex i reaktionen. Om stabiliseringen av reaktanter är större än
för ``transition-state'' kommer den effektiva aktiveringsenergin att
stiga med ökande jonstyrka och \emph{vice versa}.

Från Debye-Hückel teorin kan den s.k. ``primära kinetiska
salteffekten'' härledas. Vi utgår från ett ``transition-state''-antagande:

\begin{align}
  \label{eq:ts}
  % \ce{Fe^3+ + SCN-} & \ce{<=> [ Fe \dotsb SCN ]^2+ -> FeSCN^2+} \\
  \ce{A + B} & \ce{<=> TS -> P} \\
  \label{eq:dpdt-propto}
  \frac{d[\ce{P}]}{dt} &\propto [TS]
\end{align}

vår hastighetskonstant $k$ för elementarreaktionen:
\begin{equation}
  \ce{A + B ->[k] P}
\end{equation}
måste inkludera alla hastighetsförändrande effekter utöver beroendet på
respektive reaktants koncentration:

\begin{equation}
  \label{eq:dpdt-explicit}
  \frac{d[\ce{P}]}{dt} = k[A][B]
\end{equation}

två parametrar som kan ha sådana effekter kan vara temperatur (som
påverkar $K^\ddagger$) och jonstyrka (som påverkar \ce{\{A\}}, \ce{\{B\}} och \ce{\{TS\}})\footnote{
  aktiviteten för ett ämne brukar betecknas med klammerparentes:
  $ \ce{\{A\}} = \gamma_\ce{A}\frac{[\ce{A}]}{c^\plimsoll} $
där $\gamma_\ce{A}$ är aktivitetskoefficienten för A och beror på
lösningens sammansättning, tryck och temperatur.
}. Steady-state
halten av vårt transition-state ges av:

\begin{equation}
  \label{eq:ts-conc}
  [\ce{TS}] = \frac{\gamma_A\gamma_B}{\gamma_{\ce{TS}}}\frac{[A][B]}{c^\plimsoll}K^\ddagger
\end{equation}

där $\gamma$ är aktivitetskoefficienterna för respektive ämne. För joner
i vattenlösning kan aktivitetskoefficienten uppskattas från Debye-Hückel
teorin, i denna laboration skall vi använda oss av Debye-Hückels
gränslag\footnote{
  Debye-Hückels gränslag är endast giltig i mycket utspädda lösningar ($I
  < \SI{1e-3}{\mole\per\kg}$) men för att hålla komplexiteten av analysen
  på en rimlig nivå använder vi ändå detta uttryck i denna laboration.
}:

\begin{equation}
  \label{eq:activity}
  \log_{10}\gamma_i = -A_{DH}z_i^2\sqrt{\frac{I}{I^\plimsoll}}
\end{equation}

där $z_i$ är laddningstalet för ämnet $i$, $I^\plimsoll$ är
standard-jonstyrkan (konventionellt \SI{1}{\mole\per\kg}) och $A_{DH}$
är en (enhetslös) konstant från Debye-Hückel teorin:

\begin{equation}
  \label{eq:debye-huckel-a}
  A_{DH} = \frac{1}{\ln 10} \frac{F^3}{4\pi N_A}\sqrt{\frac{\rho b^\plimsoll}{2(\varepsilon_0\varepsilon_rRT)^3}}
\end{equation}

$F$ är Faradays konstant, $N_A$ Avogadros tal, $\rho$ lösningsmedlets
densitet, $b^{\plimsoll}$ standard-molaliteten, $\varepsilon_r$ den
relativa permittiviteten för mediumet, $\varepsilon_0$ permittiviteten för
vakuum, $R$ allmänna gaskonstanten och $T$ är temperaturen. För vatten vid
\SI{298.15}{\kelvin} är $A_{DH} = 0.510$.

Logaritmering av \cref{eq:ts-conc} och insättning av \cref{eq:activity}
ger:

\begin{align}
  \log_{10}([\ce{TS}] / c^\plimsoll) &= \log_{10}\left(
    \frac{[A][B]}{(c^\plimsoll)^2}K \right) - \left(z_A^2 + z_B^2 - (z_A+z_B)^2\right) A_{DH}
    \sqrt{\frac{I}{I^\plimsoll}} \\
  \label{eq:log-ts-corr}
  \log_{10}([\ce{TS}] / c^\plimsoll) &= \log_{10}
    \left(\frac{[A][B]}{(c^\plimsoll)^2}K
    \right) + 2z_Az_BA_{DH}\sqrt{\frac{I}{I^\plimsoll}}
\end{align}

där vi antagit att laddningen för vårt transition state är summan av
laddningarna för våra reaktanter.
Från \cref{eq:dpdt-propto,eq:dpdt-explicit,eq:log-ts-corr} ser vi att jonstyrkeeffekten
på vår observerade hastighetskonstant har följande samband:

\begin{align}
  \label{eq:kinetic-salt}
  \boxed{
  \log_{10} \left( \frac{k_{obs}}{k^{\circ}} \right) = 2 A_{DH} z_A z_B \sqrt{\frac{I}{I^\plimsoll}}
  }
\end{align}

där $k_{obs}$ är den observerade hastighetskonstanten och $k^{\circ}$ är
hastighetskonstanten extrapolerad till $I = \SI{0}{\mole\per\kg}$ (ett
förfarande som gör värden från olika experiment jämförbara med
varandra)\footnote{
 De extrapolerade hastighetskonstanterna ($k^\circ$) för obefintlig
 jonstyrka skall reproducera jämviktskonstanten: $K =
 \frac{k^\circ_{f}}{k^\circ_{b}}$.
}. Därför är det viktigt att man rapporterar vid vilken jonstyrka
mätningarna av en hastighetskonstant gjorts, och förslagsvis anger man
också ett korrigerat $k^\circ$-värde (där man också anger att man använt
-- i detta fall -- Debye-Hückels gränslag för korrigeringen).

\subsection{Förenklingar}
Vi har endast beaktat reaktionen för bildning av
monotiocyanatojärn(III). I verkligheten förekommer även komplexen givna i
\cref{tab:equilibria}.

\begin{table}
  \centering
  \footnotesize
  \caption{Några utvalda relevanta jämvikter.}
  \begin{tabular}{r@{}c@{}rcl@{}c@{}rrc@{}Sc}
    \ce{Fe3+}    & + & 2 \ce{SCN-} & $\rightleftharpoons$ & \ce{Fe(SCN)_2^+} &   &          & $\log_{10}(\beta_2)$   &=&  3.34  +- 0.02   & Ref.\cite{bahta_critical_1997} \\
    \ce{Fe3+} & + & 3 \ce{SCN-} & $\rightleftharpoons$ & \ce{Fe(SCN)3}    &   &          & $\log_{10}(\beta_3)$   &=&  3.82  +- 0.09   & Ref.\cite{bahta_critical_1997} \\
    \ce{Fe^3+}       & + & \ce{H2O}  & $\rightleftharpoons$ & \ce{FeOH^2+}     & + & \ce{H+}  & $\log_{10}(\beta_1)$   &=& -2.774 +- 0.005  & Ref.\cite{peintler_improved_2000} \\
    \ce{2Fe^3+}      & + & \ce{2H2O} & $\rightleftharpoons$ & \ce{Fe2(OH)2^4+} & + & \ce{2H+} & $\log_{10}(\beta_{1'})$ &=& -2.81  +- 0.02   & Ref.\cite{peintler_improved_2000} \\
                     &   & \ce{SCNH} & $\rightleftharpoons$ & \ce{SCN-}        & + & \ce{H+}  & pKa                   &=& -1.28            & Ref.\cite{chiang_determination_2000}
  \end{tabular}
  \label{tab:equilibria}
\end{table}



De två stabilitetskonstanterna för tiocyanatokomplexen gäller för
$I=\SI{1}{\mol\per\kg}$ medan de två för hydroxokomplexen gäller för
$I=\SI{0.1}{\mol\per\kg}$. Isotiocyansyras pKa är extrapolerad till
$I=\SI{0}{\mole\per\kg}$).

Hur stora effekterna är från dessa jämvikter beror på
$[\ce{Fe^{3+}}]_{tot}$, $[\ce{SCN-}]_{tot}$ och pH. Ni bör därför ta
detta i beaktande när ni väljer era koncentrationer.


%%% Local Variables:
%%% mode: latex
%%% TeX-master: "../main"
%%% ispell-local-dictionary: "swedish"
%%% End:
