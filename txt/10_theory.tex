\section{Teori}
\label{sec:teori}
I denna laboration skall ni studera den bimolekylära reaktionen mellan
järn(III) joner och thiocyanat joner som bilder thiocyanatojärn(II):

\begin{center}
\begin{tabular}{cc}
  \ce{Fe^3+ + SCN- <=>>[k_f][k_b] FeSCN^2+} & K=\SI{0.0}{\per\Molar}
\end{tabular}
\end{center}

\ce{FeSCN^2+} är starkt rödfärgat ($\lambda_{max}=\SI{450}{\nm}$) medan de andra reaktanterna är
färglösa. (\ce{Fe^3+} har en extinktions koefficient på XXX \si{\per\Molar\per\cm}).

Spektrofotometern registrerar transmissionen ($T = I/I_0$) vilket omvanldas till absorbans enligt:

\begin{equation}
  \label{eq:lambert-beer}
  A = -\log_{10}(\frac{I}{I_0})
\end{equation}

Lambert-Beer's lag visar att absobansen är linjärt beroende av koncentrationen av
det absorberande specien (det färgade ämnet).

\begin{equation}
  \label{eq:lambert-beer}
  A = \epsilon c l
\end{equation}

där $A$ är absorbansen, $\epsilon$ extinktionskoefficienten vid den valda våglängden och $l$ kyvettlängden.

Uppgifterna som skall lösas i denna laboration är:
\begin{enumerate}
\item $k_f$ för 8 olika temperaturer
\item Från temperaturberoendet för $k_f$ bestämma frekvensfaktorn och
  aktiverings energin enligt Arrhenius ekvation.
\item Frivillig extra uppgift: istället för aktiverings energin kan vi försöka 
  uppskatta $\Delta H^{\ddag}$ och $\Delta S^{\ddag}$ från transition state teorin
  (se bilaga X).
\end{enumerate}

\begin{equation}
  \label{eq:scn-rate}
  \frac{d[\ce{SCN-}]}{dt} = k_b[\ce{FeSCN^2+}] - k_f[\ce{Fe^3+}][\ce{SCN-}]
\end{equation}

Frågan är nu hur man skall bestämma $k_f$. För det första behöver vi ett funktionsuttryck
för [\ce{FeSCN^2+}]. Enklast är kanske i detta skede konstatera att under antagandet att
vi inte har någon mängd \ce{FeSCN^2+} vid reaktionens start gäller:

\begin{equation}
  \label{eq:fescn-scn-rel}
  [\ce{FeSCN^2+}] = [\ce{SCN-}]_0 - [\ce{SCN-}]
\end{equation}

vilket låter oss fokusera på att lösa \cref{eq:scn-rate}.

Detta kan göras med olika grad av förenklingar. Låt oss studera dem nedan.

\subsection{Pseudo första ordningens reaktion}
Jämviktskonstanten visar att den ``framåtgående'' reaktionen (bildandet av \ce{FeSCN^2+})
är den dominerande. Vidare har vi valt att låta $[\ce{Fe^3+}]_0=10[\ce{SCN-}]_0$ vilket
betyder att koncentrationen järn(III) är ganska konstant under reaktionens gång. Om vi
utnyttjar dessa antaganden och förenklar \cref{eq:scn-rate} till:

\begin{equation}
  \label{eq:scn-pseudo-rate}
  \frac{d[\ce{SCN-}]}{dt} = -k_f[\ce{Fe^3+}][\ce{SCN-}] = -k'[\ce{SCN-}] \\
\end{equation}

där $k' = k_f[\ce{Fe^3+}]$. \cref{eq:scn-rate} har en enkel lösning:

\begin{align}
  \int_{[\ce{SCN-}]_0}^{[\ce{SCN-}]_{p}} \frac{dx}{x} &= -k'\int_0^{t}d\tau \\
  \Big[ \log{x} \Big]_{[\ce{SCN-}]_0}^{[\ce{SCN-}]_{p}} &= -k't \\
  \log{[\ce{SCN-}]_{p}} - \log{[\ce{SCN-}]_{0}} &= -k't\\
  [\ce{SCN-}]_{p} &= [\ce{SCN-}]_0e^{-k't}
\end{align}

där $[\ce{SCN-}]_{p}$ betecknar pseudo-första ordningens approximation
av thiocyanat jonskoncentrationen vid tiden $t$, vilket från \cref{eq:fescn-scn-rel}
ger:

\begin{equation}
  [\ce{FeSCN^2+}]_{p}=[\ce{SCN-}]_0(1 - e^{-k't})
\end{equation}

Se bilaga \# för ett räkne exempel där vi manuellt beräknar k' med linjär regression
från absorbans data.

\subsection{Irreversibel bimolekylär kinetik}
En mer korrekt behandling är att beakta tidsberoendet av [\ce{Fe^3+}].
För enkelhetens skull betraktar vi fortfarande endast den framåt gående reaktionen.
För läslighetens skull betecknar vi från och med nu [\ce{FeSCN^2+}], [\ce{Fe^3+}] 
och [\ce{SCN-}] och  med variablerna $x$, $y$ och $z$ och de initiala koncentrationerna
med respektive versaler $X$, $Y$ och $Z$. Vi kan då beskriva koncentrationerna vid
tiden $t$ enligt:
\begin{center}
\begin{tabular}{rccccc}
        & \ce{Fe^3+} & + & \ce{SCN-} & \ce{<=>>[k_f][k_b]} & \ce{FeSCN^2+} \\
  $t=0$ &     $Y$    &   &    $Z$    &                    &      $X$        \\
  $t>0$ &  $y=Y+X-x$ &   & $z=Z+X-x$ &                    &      $x$        \\
  $t>0, X=0$ &  $y=Y+x$ &   & $z=Z-x$ &                    &      $x$        \\
\end{tabular}
\end{center}

härledningarna nedan baseras - för läslighets skull - på sista raden, $X=0$.
Vi ser att vi endast har en beroende variabel och \cref{eq:scn-rate} ger vid
försummandet av bakåtreaktionen följande integral:

\begin{align}
  \frac{d}{d t} x = k_{f} \left(Y - x\right) \left(Z - x\right) \\
  \int_{0}^{x} \frac{1}{\left(Y - \chi\right) \left(Z - \chi\right)}\, d\chi = \int_{0}^{t} k_{f}\, d\tau
\end{align}
% \begin{align}
%   \int_0^x \frac{d\chi}{(Y-\chi)(Z-\chi)} &= k_f\int_0^{t}d\tau \\
%   \frac{1}{Y-Z} \Big[ \log{\left( \frac{Y-\chi}{Z-\chi} \right)} \Big]_0^x &= k_ft
% \end{align}

nästa steg är att slå upp integralen i exmepelvis BETA Handbook of
Mathematics, lös för hand med partialbråks uppdelning  eller använda
ett CAS\footnote{  Computer Algebra System - exempelvis Mathematica,
  Maple, Maxima, SymPy m.fl.}. Friställandet av vår beroende variabel $x$
till ett explicit uttryck i den oberoende variabeln $t$ blir: 

\begin{align}
  \frac{1}{Y - Z} \left(\log{\left (\frac{Z}{Y} \right )} + \log{\left (Y - x \right )} - \log{\left (Z - x \right )}\right) = k_{f} t \\
  x = \frac{Y \left(- e^{k_{f} t \left(- Y + Z\right)} + 1\right)}{\frac{Y}{Z} - e^{k_{f} t \left(- Y + Z\right)}}
\end{align}
% \begin{align}
%   \log{\left( \frac{Y-x}{Z-x} \right)} &= k_ft(Y-Z) + \log{\left( \frac{Y}{Z} \right)} \\
%   \frac{Y-x}{Z-x} &= \frac{Y}{Z}\underbrace{e^{k_ft(Y-Z)}}_{=C} \\
%   Z(Y-x) &= (Z-x)YC \\
%   (YC-Z)x &= ZY(C-1) \\
%   x &= \frac{ZY(C-1)}{YC-Z} \\
%   x &= \frac{ZY(e^{k_ft(Y-Z)}-1)}{Ye^{k_ft(Y-Z)}-Z}
% \end{align}

\subsection{Reversibel bimolekylär kinetik}

\begin{align}
  \frac{d}{d t} x = - k_{b} x + k_{f} \left(Y - x\right) \left(Z - x\right) \\
  \int_{0}^{x} \frac{1}{- \chi k_{b} + k_{f} \left(Y - \chi\right) \left(Z - \chi\right)}\, d\chi = \int_{0}^{t} 1\, d\tau
\end{align}
\begin{align}
  \int \frac{1}{a x^{2} + b x + c}\, dx = \begin{cases} C + \frac{1}{\sqrt{- 4 a c + b^{2}}} \log{\left (\frac{2 a x + b - \sqrt{- 4 a c + b^{2}}}{2 a x + b + \sqrt{- 4 a c + b^{2}}} \right )} & \text{for}\: 4 a c < b^{2} \\C - \frac{2}{2 a x + b} & \text{for}\: 4 a c = b^{2} \\C + \frac{2}{\sqrt{4 a c - b^{2}}} \operatorname{atan}{\left (\frac{2 a x + b}{\sqrt{4 a c - b^{2}}} \right )} & \text{for}\: 4 a c > b^{2} \end{cases}
\end{align}
\begin{align}
  \begin{Bmatrix}a : k_{f}, & b : - Y k_{f} - Z k_{f} - k_{b}, & c : Y Z k_{f}\end{Bmatrix} \\
  P = \sqrt{- 4 a c + b^{2}} \\
  \frac{1}{P} \left(- \log{\left (\frac{- P + b}{P + b} \right )} + \log{\left (\frac{- P + 2 a x + b}{P + 2 a x + b} \right )}\right) = t \\
  x = \frac{\left(P - b\right) \left(- P e^{P t} + P - b e^{P t} + b\right)}{2 a \left(P + b + \left(P - b\right) e^{P t}\right)} \\
  Q = P + b \\
  R = P - b \\
  x = - \frac{Q \left(e^{P t} - 1\right)}{2 a \left(\frac{Q}{R} + e^{P t}\right)}
\end{align}


%%% Local Variables:
%%% mode: latex
%%% TeX-master: "../main"
%%% End:
